\documentclass[utf8]{article}

\usepackage[utf8]{inputenc}

\usepackage[parfill]{parskip}

\usepackage{amsmath}
\usepackage{amssymb}
\usepackage{amsfonts}
\usepackage{graphicx}
\usepackage{float}
\usepackage{listingsutf8}

\usepackage{fullpage}
\usepackage{hyperref}
\usepackage{url}
\usepackage{color}
\usepackage{listings}
\usepackage{caption}
\usepackage{subcaption}
\usepackage{enumitem}

% -----------------------------------------------------

\begin{document}

\begin{titlepage}
    \centering
    
    % Titre en haut de la page
    \vspace*{1cm}
    {\huge \bfseries INFO F-202 : Arkanoid\\
                    Rapport \par}
    
    % Espace vertical pour centrer le logo
    \vfill
    
    % Logo au milieu de la page
    \begin{figure}[h]
        \centering
        \includegraphics[scale=0.2]{images/logo.png}
    \end{figure}
    
    % Espace vertical pour descendre l'auteur et la date en bas
    \vfill
    
    % Auteur et date en bas de la page
    {\large Auteur: Rocca Manuel\\ 
            Matricule: 000596086 \\ 
            Section: INFO \par}
    {\large \today \par}
\end{titlepage}

\newpage
\tableofcontents

\newpage

% -----------------------------------------------------

\section{Introduction}
Dans le cadre de notre cours de Bases de Données, nous avons été amenés à réaliser un projet consistant en
la réalisation d’un système de gestion d’inventaire pour un jeu de rôle. Celui-ci, est séparé en deux phases.
La première consiste en la réalisation d’un diagramme Entité-Association modélisant le projet et ses contraintes, 
un modèle relationnel correspondant au diagramme et indiquant les contraintes d’intégrité et un rapport 
justifiant nos hypothèses et justifications concernant les diagrammes.
Le projet est à réaliser en groupe de 4 personnes. Ce travail, avait comme objectif de
nous faire manipuler les concepts vus au cours tels que les clés, les contraintes d’intégrité, 
les diagrammes d'entité-Association, etc et avec comme finalité de la première phase, un modèle correct 
et cohérent représentant un RPG.
Ce devoir, nous aura permis de renforcer nos compétences en bases de données, tout en travaillant sur un projet concret
afin de préparer notre examen en y appliquant les concepts appris au cours.



\subsection{Contraintes d'intégrité}
Chaque Joueur.Pseudo doit être unique. 
Tout Joueur.Lvl, Joueur.Xp et Joueurs.Or doivent être des entiers positifs.
Un joueur doit créer au moins un personnage lors de son inscription. \\

Chaque Personnage.ID_personnage est unique.
Personnage.Vie et Personnage.Mana doivent être des entiers positifs.
Personnage.Force, Personnage.Agilité et Personnage.Intelligence peuvent être négatifs et donneront un débuff au joueur en fonction de la stat (ratio \%)
Personnage.Classe doit appartenir à {guerrier, mage, voleur}.
Un Personnage peut attaquer plusieurs Monstre à la fois (s'il possède une attaque AOE "Area of Effect")\\

Inventaire.capacité doit être un entier positif.
Les interactions du personnage avec son inventaire sont exclusivement {collecter, équiper, jeter} et rien d'autre
Un inventaire ne peut pas contenir plus d'objets que sa capacité le permet.\\

Objet.Prix doit être un entier positif. 
Plusieurs objets de même type et de même nom peuvent avoir des statistiques différentes et donc être différents (Objet.ID_objet différent).
Plusieurs personnages peuvent détenir le même objet dans leur inventaire (même Objet.ID_objet)
Potion.Soins, Armure.Défense et Arme.Puissance sont des entiers strictement positifs.\\

Les sorts sont uniques selon la classe de personnage.
Sort.Coût_mana et Sort.Puissance_attaque doivent être des entiers positifs.
Si un personnage améliore un sort, un nouveau sort du même nom est créé dans Sort avec un ID_sort différent.\\ 

Monstre.Attaque, Monstre.Défense et Monstre.Vie doivent être des entiers Positifs. 
La récompense d'un monstre doit toujours comprendre au moins un objet avec une quantité prédéfinie positive et une probabilité comprise entre 0\% et 100\%.
Un Monstre peut attaquer plusieurs Personnages à la fois (s'il possède une attaque AOE "Area of Effect").\\

Quête.Nom peut être identique pour plusieurs Quête.
Quête.Description ne peut pas être vide.
Quête.Difficulté doit être un entier positif.
Quête.Or et Quête.XP doivent être des entiers positifs.
Une Quête n'est pas forcément donnée par un Pnj, elle peut être trouvée dans le jeu.\\

Pnj.Nom peut être identique pour plusieurs Pnj.
Un Pnj peut discuter avec plusieurs Personnages en même temps.
Pnj.Dialog ne peut pas être vide, il reste identique pour tous les personnages rencontrés.
Si le Pnj vend des Objet, ceux-ci ne changeront pas.


\section{Modèle Relationnel}

\section{Hypothèses et justifications}
Voila nos hypothèses et justifications pour le diagramme Entité-Association.
\subsection{Hypothèses}
\begin{itemize}
    \item Les classes sont un attribut de personnage.
    \item Les armes, les armures, les potions, les artefacts, héritent d'objet.
    \item La table PNJ a comme attribut un ou plusieurs dialogues. Ceci permet de ne pas trop complexifier le diagramme.
    \item Nous avons mis un ID aux monstres puisqu'en fonction de ses statistiques, ceux-ci sont différents pour un même nom.
    \item Nous avons mis un ID aux PNJ afin de ne pas avoir de problèmes s'ils ont le même nom.
    \item Nous avons mis un ID aux objets pour les différencier.
    \item Nous avons mis un ID aux sorts pour les différencier.
\end{itemize}
\subsection{Justifications}
\begin{itemize}
    \item Chaque personnage se spécialise avec sa classe, un héritage aurait pu être possible mais nous avons préféré un attribut pour simplifier le modèle.
    \item Un objet est forcément une arme, une armure, une potion ou un artefact. Donc une arme, une armure, une potion, un artefact est un objet.
    \item Un PNJ obligatoirement un dialogue pour pouvoir interagir avec le joueur un minimum. L'attribut dialogue permet de ne pas devoir créer une table dialogue.
    \item Le monstre est unique en fonction de ses statistiques, il est donc nécessaire de lui attribuer un ID pour les différencier.
    \item Le PNJ est unique en fonction de son nom, il est donc nécessaire de lui attribuer un ID pour les différencier.
    \item Un objet est unique en fonction de son id mais pas de son nom, ils pourraient avoir des statistiques différentes pour un même nom.
    \item Un sort est unique en fonction de son id mais pas de son nom, ils pourraient avoir des statistiques différentes pour un même nom.
\end{itemize}

\section{Conclusion}
Ce projet nous a permis de mettre en pratique les concepts vus au cours et de nous familiariser avec la modélisation de bases de données.
Nous avons également appris à travailler en groupe et à nous répartir les tâches de manière efficace pour réaliser de manière correcte cette première phase du projet.
Finalement, nous avons réussi à atteindre les objectifs fixés au début du projet.
En conclusion, ce projet a été une expérience très enrichissante pour nous. Il nous a permis de développer nos compétences 
en tant que futurs informaticiens et manipulateurs de bases de données, de renforcer notre capacité à travailler en équipe et de produire un travail
de qualité. Nous sommes fiers des efforts prodigués et avons hâte de retravailler ensemble et de pouvoir 
découvrir de nouvelles solutions à des problèmes ensemble au cours de la phase 2.

\end{document}
